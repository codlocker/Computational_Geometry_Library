\subsection*{Documentation for the Algorithms used }

\begin{quote}
{\bfseries Objective}\+: In this assignment, you have to implement three different convex hull algorithms as discussed in class. The code should be written such that it provides an A\+PI for others to interact with your code. Design your code properly. It will be good if you write your code in C++. If you want to use any other programming language then discuss with I/C. The code should be well documented, commented, and indented. \end{quote}



\begin{DoxyEnumerate}
\item \href{https://en.wikipedia.org/wiki/Graham_scan}{\tt Graham Scans Algorithm} \begin{quote}
Graham\textquotesingle{}s scan is a method of finding the convex hull of a finite set of points in the plane with time complexity O(n log n). It is named after Ronald Graham, who published the original algorithm in 1972. The algorithm finds all vertices of the convex hull ordered along its boundary. It uses a stack to detect and remove concavities in the boundary efficiently. \end{quote}

\item \mbox{[}Andrews Algorithm\mbox{]}(\href{http://www.codecodex.com/wiki/Andrew's_Monotone_Chain_Algorithm}{\tt http\+://www.\+codecodex.\+com/wiki/\+Andrew\textquotesingle{}s\+\_\+\+Monotone\+\_\+\+Chain\+\_\+\+Algorithm}) \begin{quote}
It does so by first sorting the points lexicographically (first by x-\/coordinate, and in case of a tie, by y-\/coordinate), and then constructing upper and lower hulls of the points in O(n) time. An upper hull is the part of the convex hull, which is visible from the above. It runs from its rightmost point to the leftmost point in counterclockwise order. Lower hull is the remaining part of the convex hull. \end{quote}

\item \href{http://www.personal.kent.edu/~rmuhamma/Compgeometry/MyCG/ConvexHull/jarvisMarch.htm}{\tt Jarvis Mach Algorithm} \begin{quote}
Jarvis march computes the C\+H(\+Q) by a technique known as gift wrapping or package wrapping.\end{quote}

\end{DoxyEnumerate}